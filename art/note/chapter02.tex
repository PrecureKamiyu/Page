% Created 2023-06-20 Tue 15:24
% Intended LaTeX compiler: pdflatex
\documentclass[11pt]{article}
\usepackage[utf8]{inputenc}
\usepackage[T1]{fontenc}
\usepackage{graphicx}
\usepackage{grffile}
\usepackage{longtable}
\usepackage{wrapfig}
\usepackage{rotating}
\usepackage[normalem]{ulem}
\usepackage{amsmath}
\usepackage{textcomp}
\usepackage{amssymb}
\usepackage{capt-of}
\usepackage{hyperref}
\date{\today}
\title{}
\hypersetup{
 pdfauthor={},
 pdftitle={},
 pdfkeywords={},
 pdfsubject={},
 pdfcreator={Emacs 26.3 (Org mode 9.1.9)}, 
 pdflang={English}}
\begin{document}

\tableofcontents

\section{chapter 2 ISA and risc-v instruction}
\label{sec:org6d48388}
\subsection{ISA 是什么?}
\label{sec:orgb6df826}

risc-v 是一种指令集架构: ISA (instruction set architecture). 也称为处理器架构

使用同一个指令集架构的计算机就是这种 ISA 的系列机, 比如说 risc-v 系列机, Arm 系列机.  


一个 ISA 有很多参数, 最基本的参数就是通用寄存器的位宽, 一般有32bit和64
bit之分. 位宽, 指定了通用寄存器的宽度. 其决定了决定了寻址范围的大小, 数
据运算能力的强弱. 地址长度为寄存器的长度, 64 bit 的地址范围当然是比 32
bit 的要长的多. 

其需要和指令编码长度分开, 指令的编码长度是越小越好的

\subsubsection{原则}
\label{sec:org039fef6}

\begin{enumerate}
\item 简单性来自于规则性
\item 越小越快
\item 加速经常性时间.  经常使用的指令尽量短.
\item 需要良好的折衷.
\end{enumerate}


\subsubsection{汇编语言的优缺点}
\label{sec:org372c34d}

废话吧这是

\subsection{risc-v ISA}
\label{sec:org828b6f7}

\subsubsection{指令格式}
\label{sec:orgc590ab9}
汇编指令的格式为 

op dst, src1, src2

这是最一般的指令格式, 其中有 dst(destination), 两个来源 src. 一个指令
对应一个操作. 一行最多一条指令. C语言之中的操作会被分解为一条或者是多
条指令. 来源可以是 imm 也可以是寄存器的值. 

\subsubsection{寄存器}
\label{sec:org85acf5b}

32个通用寄存器, x0--x31. 注意这里仅仅涉及 RV32I 或者是 RV64I 的寄存器,
这两个是 risc-v 指令的子集, 其他子集会用到其他的寄存器. 这算 dark side

算术逻辑运算所操作的数据必须直接来自于寄存器\footnote{在 x86 之中并非如此, 大部分指令都可以对内存之中的数据进行操作.
比如说我们有一个 add 指令. 写为 

add x1, x1, Imm(x2)

(当然了, x1 并不是 x86 之中寄存器的名字)这是说, 我们找到 Imm + x2 为地
址的内存数据, 加上 x1 之后再加到 x1 上.}. 涉及内存的操作只有
load 指令和 save 指令. 其中 x0 是一个特殊的寄存器, 其值恒为 0. RV32I
指令集通用寄存器的长度为 32 位, 而RV64I指令集对应的长度是 64 位

汇编是相对原始的, 汇编语言之中没有变量的概念. 直接使用寄存器操作. 直接
使用寄存器能够最大化速度. 缺点在于寄存器的数量非常有限, 需要好好规划. 


\begin{center}
\begin{tabular}{lll}
\hline
寄存器 & 符号 & 说明\\
\hline
x0 & zero & 值为0\\
x1 & ra & return address\\
x2 & sp & stack pointer\\
x3 & gp & global pointer\\
x4 & tp & thread pointer\\
x5-x7 & t0-t2 & temporary register\\
x8 & s0/fp & save register/frame pointer\\
x9 & s1 & save register\\
x10-x11 & a0-a1 & return value\\
x12-x17 & a2-a7 & function argument\\
x18-x27 & s2-s11 & save register\\
x28-x31 & t3-t6 & temporary register\\
\hline
\end{tabular}
\end{center}

\subsubsection{内存}
\label{sec:org8e86807}

Save Load操作, 实现寄存器和内存之间的读写. 这里我们再次细明一下存储单
元和地址. 

一个地址单元是一个 Byte, 一个 Byte 是 8 个 bit. 我们常常用十六进制表示,那
么一个 Byte 可以使用两个字符表示. 比如说 0xFF. 一个 word 表示的是 32
个 bit, 也就是 4 个 Byte. 也有 double word (简写为 dword) 表示的是两个
word, 也就是 8 个 Byte. 还有 half word, 表示的是半个 word, 那就是16个
bit. 这些单位常常以后缀的方式出现在 save load 指令之中. 


\subsection{Instructions introduction}
\label{sec:org6c4ef60}
\subsubsection{primitive instructions}
\label{sec:org7096a0f}

\begin{itemize}
\item add 指令
\item sub 指令
\item subi 指令
\item imm 是什么
\item mul 指令
\item div 指令
\item or and 指令
\item sll sla
\item save load 指令
\item shamt
\item bne beq 指令
\item jalr jal 指令
\end{itemize}

\subsubsection{pseudo instruction}
\label{sec:org767f2a1}
伪指令有简单理解的, 比如说 mv, sw, lw (sw, lw 本身不是伪指令, 但其有能
够作为伪指令). 也有比较难搞的, 比如说 la li 伪指令.

意思是将 \texttt{Label} 所在的指令的地址传输到 \texttt{rd} 上. 其中 \texttt{Label} 表示的是当前 PC 的值和目标指令的差值, 记为 \texttt{delta}, 长度为 32 位. 

\subsection{How to write a function using assembly [0/3]}
\label{sec:org33bb453}
\begin{itemize}
\item[{$\square$}] part 1
\item[{$\square$}] part 2
\item[{$\square$}] part 3
\end{itemize}
\subsubsection{How to write a loop}
\label{sec:org640f597}

我们需要分支跳转的指令来实现这点, 比如说, 我们要实现一个简单的语句: 

for (int i = 0 ; i < 2 ; i++) \{
    ans = ans + i;
\}                 

分析一下就是
\begin{enumerate}
\item 写下一个 Label
\item ans = ans + i;
\item i++
\item 分支跳转的判断
\end{enumerate}

有: 
    add t0, x0, x0
    add t1, x0, x0
    li  t2, 2
Loop:
    add t1, t1, t0
    addi t0, t0, 1
    bne t0, t2, Loop

大概这样, t0 是 i; t1 是 ans; t2 是 2. 当然啦, bne 换为 bge 或许更好.
总是上面就是一个简单的 Loop. 

\subsubsection{the concept of basic block}
\label{sec:orgd921fc7}

一段经常性执行的代码便是一个 basic block. 由于加速经常性时间的构想, 这
段代码实际上会被特地的优化, 使得其运行速度变得更快. 

\subsubsection{Function and stack}
\label{sec:orga3b4667}

我们有的时候要将数据存入 stack, 而 stack 是位于内存里面的. 我们在手动
进行汇编程序的编写的时候, 要初始化数据, 这个时候我们可以手动地为这些数
据分配空间, 放在 stack 里面. 我们接下来给出一个 bing 的例子:


factorial:
    addi sp, sp, -16  \# Allocate space on stack
    sw ra, 12(sp)     \# Save return address
    sw a0, 8(sp)      \# Save argument n
    addi a1, x0, 1    \# Initialize result to 1
    beq a0, x0, end   \# If n == 0, jump to end
loop:
    mul a1, a1, a0    \# result *= n
    addi a0, a0, -1   \# n--
    bne a0, x0, loop  \# If n != 0, jump to loop
end:
    mv a0, a1         \# Return result in a0
    lw ra, 12(sp)     \# Restore return address
    addi sp, sp, 16   \# Deallocate space on stack
    ret               \# Return from function

\subsection{The expression of an instruction}
\label{sec:org22b9fc6}

\subsubsection{the field of an instruction code}
\label{sec:org8edcda3}

一个 instruction code 是32位. 一般来说会划分出大约5到6个field, 这些
field有它们自己的功能. 对于不同类型的指令码, 这些field各有不同. 我们这
里用 add 指令为例子.

\begin{center}
\begin{tabular}{lrrrrrr}
\hline
field & funct7 & rst2 & rst1 & funct3 & rd & opcode\\
\hline
bits & 7 & 5 & 5 & 3 & 5 & 7\\
\hline
\end{tabular}
\end{center}

这是 R-type 指令. 

\subsubsection{the type of the instruction code}
\label{sec:org6f9e2e4}

共有很多类型, 常见的有 R, I, J, S, SB, U 型指令. 



\section{imm}
\label{sec:org9b81967}
操作名后面加上了一个 i, 那么这个操作是立即数操作. 
其将第二个操作数当作是立即数. 
需要知道的是, 立即数会进行一个符号扩展, 扩展为一个补码表示的.
e.g. \texttt{addi x1, x1, 5}

随后, 更值得注意的是, 存在 \texttt{addi}, 但是不存在 \texttt{subi}. 
\texttt{f = g - 10}
实际上是 \texttt{addi x3, x4, -10}
\%EndOfParagraph

\section{mul}
\label{sec:org674003f}
mulh 取高位指令. 详情请看 ppt 
\%EndOfParagraph

\section{div}
\label{sec:org997dbff}
详情请看 ppt. 
\%EndOfParagraph

\section{or and and}
\label{sec:orgff5a162}
值得注意的是, risc-v 之中并没有取非操作. 
但是我们可以使用前面的知识, 进行代替. 

\section{shift}
\label{sec:orgae63253}

值得注意的是, 没有算术左移对应的指令
在一定范围内, 也就是算术左移没有发生错误的时候, 算术左移和逻辑左移是等价的. 这一点看以前的记录. 

我们复习一下, 什么时候算术位移会发生错误. 我们有这样一个判断标准: 对于左移, 如果左移之后再右移, 不能回到原本的数字的话, 
那么这个左移就出错了. 于是说, 对于左移, 正数丢弃了 \(1\), 或者是负数丢弃了 \(0\), 就会出错.

\section{save load}
\label{sec:org8e2937b}

我们可以将寄存器之中东西塞到内存之中. 我们通过内存地址进行内存的访问. 

需要注意的是, 其他指令的操作数均是寄存器之中的数, 仅有 \texttt{save load} 指令能够对内存进行操作. 
这是 risc-v 的特点之一. 你可以看出 x86 里面并不是这样的. 格式如下: 
\texttt{memop reg, offset(bAddrReg)}
解释见 ppt. \texttt{memop} 指的是内存相关的操作(也就是 \texttt{s} 或者 \texttt{l}), \texttt{reg} 指的是目标寄存器, 第二个操作数是 
\texttt{offset(bAddrReg)}, 一其中 \texttt{offset} 是一个偏移量; \texttt{bAddrReg} 是寄存器, 将寄存器里的值当作是地址. 整体的地址便是 \texttt{offset + bAddrReg}. \footnote{在 x86 里面, 内存地址的表示更为复杂, 详情请自己找}

为传输指令指定大小, 见图\textasciitilde{}\ref{tab:daxiao}
\begin{figure}
		\centering
		\begin{tabular}{|c|l|}
		\hline 
		\texttt{sw} &  取字   \\ \hline 
		\texttt{sd} &  取双字 \\ \hline 
		\texttt{sh} &  取半字 \\ \hline 
		\texttt{sb} &  取byte \\ \hline 
		\end{tabular}
		\caption{为命令指定大小}\label{tab:daxiao}
\end{figure}
\%EndOfParagraph 

\section{sign extension}
\label{sec:org2c57bff}

我们进行 \texttt{load} 命令的时候, 要指定大小, 比如说 \texttt{w}, \texttt{d}, \texttt{h}, \texttt{b}.
这就是说, 我们要将一个长度为 \(32/64/16/8\) 的数据, 送到 \(32\) 或者是 \(64\) 位的寄存器里面. 这个时候, 
计算机会对传输的数据进行 \textbf{符号扩展}. 也就是将这一小串数据, 看作是补码, 然后扩展为 \(32\) 或者\(64\)位的补码. 

比如说一个传入了一个字节 \texttt{0x80}, 写为二进制为 \texttt{10000000}. 也就是有符号位 \(1\), 是一个负数, 于是扩展为 \(32\) 位的补码的时候就变为 \texttt{0xFF80}.

值得注意的是, 默认读数据的时候, 写入的时候, 会将数据扩展, 看作是有符号的
若是需要写入无符号数的话, 需要加上 \texttt{u}. 

e.g. 传输数据的举例. 见 ppt 实际上不是很难. 

\paragraph{涉及掩码的数据传输}

\section{add}
\label{sec:orgabb8348}
add 指令的表示为 
add rd, rst1, rst2

rd is register of destination

The command tell computer to do the computation---\texttt{rd = rst1 + rst2}. Note that rst1 rst2 are treated as signed number. 


Additionally, \texttt{addi} tells computer to do \texttt{rd = rst1 + imm}, where \texttt{rst2} is replaced by an immediate number.

\section{sub}
\label{sec:org7352b56}
\centerline\{\texttt{sub, rd, rst1, rst2}\}
is the form of the instruction, telling that \texttt{rd = rst1 - rst2}. 

It is worth noting that there is no such thing as \texttt{subi}, cause \texttt{addi} can do the same thing.
\subsubsection{mul}
\section{div rem}
\label{sec:orgddcd71a}
\section{xor}
\label{sec:org96a1db5}
Logical operations can do thing likes 掩码. To achieve this we can use \texttt{and} and \texttt{0xFFFFFFFF}. Let us assume that the length of register is 32. A number \texttt{and} \texttt{0xFFFFFFFF} is the number itself. But when it \texttt{and} \texttt{0x0FFFFFFF}, the number loose it first four bits.

\section{shift}
\label{sec:orge0f4bab}
\centerline\{\texttt{s + l/r + a/l + [i]}\}
is the decompostion of an instruction,
where \texttt{s} for shift, \texttt{l,r} for \textbf{left} or \textbf{right}, \texttt{a,l} for \textbf{arithmetic} or \textbf{logical}. \texttt{[i]} is optional, which stands for \texttt{imm}.

About the arithmetic shift, you check 01.pdf out.

\section{shamt}
\label{sec:orgf531c41}
Indeed, for a 64-bits data store in a register. It would be of no use to shift of 64-bits, which resulting that in \texttt{slai} or other shifting command ending with \texttt{i}, only the lowest 6-bits of immediate number are useful. Other bits are abandoned. The remaining part is called shamt. 

\section{shift left arithmetic}
\label{sec:orgf083de1}

There is not such thing as \texttt{sla[i]}. We already know when the shift cause ailment. Exactly when the number is starting with \texttt{10} or \texttt{01} the result of \texttt{sla} is not what we want. 

However, you may check that when there is no ailment, \texttt{sla} works just like \texttt{sll}. So \texttt{sla} become less needed. 
\subsubsection{s, l}
\centerline\{\texttt{s/l r, offset(AddR)}\}
where \texttt{AddR} is a register. The command tells computer to load data from address \texttt{offset + AddR} to \texttt{r}, or to save the data in \texttt{r} to address \texttt{offset + AddR}.
\subsubsection{address and word and byte}
A \textbf{word} in risc-v has \textbf{32} bits. There arouses an interesting question: how to load 32-bit data to a 64-bit register? 

No, what I am saying is that you need to care for whether the data is unsigned type or not. You need to expand a number when it is treated as a negative number. 
\subsubsection{slt}
slt for set less than. \texttt{slt} is an instruction to compare the value of some data. 

\centerline\{\texttt{slt rd, rst1, rst2}\} 
means that \texttt{rd = rst1 and rst2}


\section{bne beq blt bltu}
\label{sec:org07ffc69}
\begin{enumerate}
\item b for break
\item eq for equal.
\item ne for not equal. 
\item lt for less then.
\item ge for greater or equal.
\end{enumerate}
Use this set of command to jump which is used to achieve if-else structure. Note that for blt and bge, there exists unsigned type of commands. 

\section{jalr jal}
\label{sec:orga6e45ba}
\centerline{jal rd, Label}
称为无条件跳转. PC+4 存贮在 rd 之中. 并且 PC 赋值为 \texttt{Label}. 

\texttt{Label} 是一个写在程序行首的标签, 比如说 \texttt{Exit} 或者 \texttt{Loop} 等. 程序运行的时候此标签会翻译为一个指令的地址. 

The actual operation it takes is PC = PC + \texttt{offSet}, where \texttt{offSet} is translated from \texttt{Label}. 

\centerline{jalr rd, offSet(AddR)}
是 \texttt{jalr} 的格式. 其表示, \texttt{rd} = PC + 4. 将 PC 的值赋为 \texttt{offset + AddR}. 
\end{document}