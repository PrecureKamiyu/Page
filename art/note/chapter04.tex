% Created 2023-04-03 Mon 00:35
% Intended LaTeX compiler: pdflatex
\documentclass[11pt]{ctexart}
\usepackage[utf8]{inputenc}
\usepackage[T1]{fontenc}
\usepackage{graphicx}
\usepackage{grffile}
\usepackage{longtable}
\usepackage{wrapfig}
\usepackage{rotating}
\usepackage[normalem]{ulem}
\usepackage{amsmath}
\usepackage{textcomp}
\usepackage{amssymb}
\usepackage{capt-of}
\usepackage{hyperref}
\date{\today}
\title{}
\hypersetup{
 pdfauthor={},
 pdftitle={},
 pdfkeywords={},
 pdfsubject={},
 pdfcreator={Emacs 26.3 (Org mode 9.1.9)}, 
 pdflang={English}}
\begin{document}

\tableofcontents

\section{第四章 cpu 的组成}
\label{sec:org497fef1}
\subsection{概述}
\label{sec:orga24b7a9}
\subsubsection{cpu 的构造法}
\label{sec:org9f4d812}
对于一个 cpu , 其中分为三个部分. 第一个部分是处理器, 第二部分是内存,
第三部分是总线.  
\begin{enumerate}
\item 处理器, 里面由 ALU, 寄存器, PC
\item 内存, 就是内存相关的, Cache 缓存, memory 内存.
\item 总线, 就是 IO 相关的
\end{enumerate}
\subsubsection{PC}
\label{sec:org9f96676}
PC 就是存储着指令, 其告诉 ALU 该如何处理寄存器. 
\subsubsection{ALU}
\label{sec:org8853663}
ALU 就是运算单元, 其能够执行 add sub 等指令, 对寄存器进行操作. 
\subsubsection{CACHE}
\label{sec:org4420c6d}
\begin{enumerate}
\item Cache 简介
\label{sec:orgcc1cbef}
Cache 是缓存, 是一种更为快速的内存. 速度和数量介于内存和寄存器之间. 

Cache 用于提高效率. 我们目前知道了三种不同效率的存储单元, 将适合的任务
分配给它们, 以此提高效率 (当然也是成本效率)
\item SRAM
\label{sec:org0e3448b}

Cache memory operates between 10 to 100 times faster than RAM,
requiring only a few nanoseconds to respond to a CPU request. The name
of the actual hardware that is used for cache memory is high-speed
static random access memory (SRAM).

实现的 Cache 的硬件为 SRAM, (high-speed static random access
memory). 而 DRAM (dynamic random access memory) 用于构建 memory.SRAM
之中的数据, 只要没有断电 (power), 就能够一直保存, 不同于 DRAM 的数据那
样, DRAM 之中的数据, 一把来说几秒钟后就会消失, 需要周期性地 refresh 一
下. SRAM use latches and transistor to store data.  
\item Cache 的多级建构
\label{sec:orgb2469a0}

Cache 分为了三个 Level, 有 L1, L2, L3, 其中 L3 速度最慢, 是用来辅助
L1, L2 的.

Cache memory is a small amount of fast memory that is used to store
frequently accessed data. It is located close to the CPU and is used
to reduce the average time it takes to access data from the main
memory. 

There are three general levels of cache: L1 cache, L2 cache, and L3
cache. Each level is differentiated by its speed, size, and proximity
to the CPU. 

L1 cache, or primary cache, is extremely fast but relatively small. It
is usually embedded in the processor chip as CPU cache³. 

L2 cache, or secondary cache, is often more capacious than L1 but
slower. It is usually located on the processor chip or on a separate
chip on the motherboard³. 

L3 cache, or tertiary cache, is larger than L2 but slower. It is
usually located on a separate chip on the motherboard³. 

Is there anything else you would like to know about cache memory?

Computers, Explained -SearchStorage. \url{https://www.techtarget.com/searchstorage/definition/cache-memory}
访问时间 2023/4/1.

(2) Explainer: L1 vs. L2 vs. L3 Cache |
TechSpot. \url{https://www.techspot.com/article/2066-cpu-l1-l2-l3-cache/} 访
问时间 2023/4/1.

(3) How Does CPU Cache Work and What Are L1, L2, and L3 Cache? -
MUO. \url{https://www.makeuseof.com/tag/what-is-cpu-cache/} 访问时间
2023/4/1.
\end{enumerate}
\subsection{interlude}
\label{sec:orgc00afac}

Havard architecture and Princeton architecture

程序有存储的地方, 数据也有存储的地方, 我们常将其抽象出来, 认为这两个部
分是存在一个地方里的. 但实际上有区别. 

数据的存储单元, 程序的存储单元, 在哈佛架构之中是分开的. 
也就是说, 在哈佛架构中, 存在两个单元, 分别由两个总线进行单元和处理器之间的数据传输
\subsection{组件的硬件构成}
\label{sec:orgc54deba}
\subsubsection{Processor 的简单构成}
\label{sec:org28bf0e3}
\begin{enumerate}
\item ALU Register PC Extender
\label{sec:org86ffe21}

\begin{enumerate}
\item ALU 是算术运算单元, 即, 进行 and or add sub 运算的单元.
\item Register 我们很熟了, 支持读写操作. PC 我们也很熟了, 支持 +4 , + imm
操作.
\item Extender 是 imm 生成单元也是扩展单元,我们有的时候需要进行一些数据的
符号扩展.
\end{enumerate}
\item ALU (算术逻辑运算单元) 的功能和硬件实现 [100\%]
\label{sec:orgd090755}

\begin{itemize}
\item[{$\boxtimes$}] 完善该标题
\end{itemize}
输入两个 32 位数据, 输出一个 32 位的数据. 进行位运算或者加减运算. 

\begin{enumerate}
\item ALU 的接线
\label{sec:orga0220ea}

一个一位 ALU 单元应该
\begin{enumerate}
\item 根据 ALU control 的值, 决定 ALU 该执行什么运算
\item 根据输入 A B 给出结果 R
\item 判断是否溢出, 接出一根线: Overflow, 其值为 1 当且仅当发生溢出
\item 判断结果是否为 0, 接出线: Zero, 其值为 1 当且仅当结果为 0
\item 接出一根线, Carry Out. 用于串联的进位
\end{enumerate}
\item ALUop 和 ALU control
\label{sec:org40573f9}

ALUop 是一个二位的信号, 其和 funct field 结合在一起, 通过 ALU 
control 单元, 生成一个 ALU control 信号. 

这里使用的是多级\footnote{这种多级译码的方式---主控制单元生成 ALUop 位作用 ALU 的输入控制
信号, 在生成实际信号来控制 ALU---是一种常见的方式. 多级控制可以减小主
控制单元的规模. 多个小的控制单元可以潜在地减小控制单元的延迟.}的控制信号生成器. 多级, 但是每一级的规模很小, 这使得信号生成的延迟降低了. 控制信号的延迟是非常重要的参数. CPU 的时钟周期就取决于这个参数. 

但是出于不明原因, 我们这里居然没有提及 ALU control 信号, 将其和 ALUop
混为一谈, 着实离谱. 下面有一些 ALUop 的出现, 他们实际上指的是 ALU
control 信号

ALUop 是指令码的一个字段, 其和 funct 字段经过 ALU 控制单元生成实际的控
制信号. 见下表:

\begin{center}
\begin{tabular}{rrrr}
ALUop & funct7 & funct3 & 操作\\
00 & XXXXXXX & XXX & 0010\\
X1 & XXXXXXX & XXX & 0110\\
1X & 0000000 & 000 & 0010\\
1X & 0100000 & 000 & 0110\\
1X & 0000000 & 111 & 0000\\
1X & 0000000 & 110 & 0001\\
\end{tabular}
\end{center}

能够看出, 实际上控制单元的输入可以简化, 比如说
\item ALU control 信号的功能
\label{sec:org8a4e9fb}

\begin{center}
\begin{tabular}{rll}
ALU con & 操作 & \\
0000 & and & \\
0001 & or & \\
0010 & add & \\
0110 & sub & substraction\\
0111 & slt & set on less than\\
1100 & nor & \\
\end{tabular}
\end{center}

\item 构建简单的 ALU
\label{sec:org0b66fab}

以 Multiplexer 为基础, 而后构建 and or add 操作
and 使用 and 门, or 使用 or 门, add 使用一个一位 Full adder. 
构建是简单的.

ALU con 的后两位是用于 Multiplexer 的
\begin{itemize}
\item 00 代表 Multiplexer 的第 0 个输入: A and B
\item 01 代表 Multiplexer 的第 1 个输入: A or  B
\item 10 代表 Multiplexer 的第 2 个输入: A + (B)
\item 11 代表 Multiplexer 的第 3 个输入: Less
\end{itemize}

其中 (B) 代表对 B 进行一定处理之后的数据. Less 是 slt 的实现之中会稍微
提及的一个信号. 

\item one bit ALUs 的串联 [5/5]
\label{sec:orgb45f552}

在串联之中我们要实现
\begin{itemize}
\item[{$\boxtimes$}] sub 操作
\item[{$\boxtimes$}] slt 操作
\item[{$\boxtimes$}] nor 操作
\item[{$\boxtimes$}] Overflow 判断
\item[{$\boxtimes$}] Zero 判断
\end{itemize}

\begin{enumerate}
\item sub 操作
\label{sec:org2f7535b}

设 ALUop 的第三位为 Binvert. 通过等式 R = A - B = A + B'来实现减法.

\begin{enumerate}
\item 在 one bit ALU 之中, 通过一个 Mux 和反相器, 使 B 取反
\item 接入末位 ALU viz., ALU0 的 CarryIn. 使得结果 +1
\end{enumerate}

这就有 R = A+B'+1. 也就有 R = A-B. 

\item slt 操作
\label{sec:orgcc635ce}

我们接入 Less 信号, 作为 ALU 的输入, 这是当然的. Less 信号很特殊, 他在
one bit ALU 之中直接输出, 并且整体作为 Mux 的最后一个输入. 因为 slt 的
结果 R 比较特殊, 只有两个取值: 1 和 0; 即, 说除了末位, 所有位的值
为 0. Less 也是如此. 而对于末位, 只需要将 A - B 的结果的符号位塞进去就
行了. 设结果为 R 

R = (A < B) 

true 代表 1, false 代表 0. 我们用 Less 作为 one bit ALU 的输入信号. 
我们只需要计算出 A - B 的值, 然后 A - B 的符号传回 Less 的末位 (Less
在其他位的值均为0), 最后 result 直接等于 Less.  

\item nor 操作
\label{sec:org792d547}

ALUop = 1100 的时候, 其为 ALU 为 nor 操作. 观察后两位, 这个时候
Multiplexer 选择第 0 位数据理论上进行的是 and 操作, 只需要让 ALUop 的
最高位为 Ainvert. 其为 1 的时候, A 的数据反相. 由于

$$
\overline {A + B} = \overline A * \overline B
$$

就有 \(R = \overline{A + B}\)

\item Overflow 判断
\label{sec:org0cf0a6b}

设输入的两个符号位为 s1 s2 , 结果的符号位为 s3, 那么有

Overflow = (s1 and s2) xor s3

就有, 当溢出发生的时候, Overflow 为 1.

\item Zero 判断
\label{sec:org7765e4c}

每一位结果取 nor 即为结果.
\end{enumerate}
\end{enumerate}

\item PC
\label{sec:orga7f3886}

PC 是一个寄存器, 存储着当前指令的地址. 当当前指令执行完毕之后, 
PC = PC + 4, 其值指向下一条指令. 

并且, 在 SB 型指令, viz., 分支跳转的指令 (比如说 bne) 执行的过程之中, 
PC 还有可能变为 PC + offset. 

那么 PC 应
\begin{enumerate}
\item PC 能够变为 PC + 4
\item PC 能够变为 PC + offset, 其中 offset 是来自立即数产生器的.
\end{enumerate}

那么 PC 应该有一个控制信号, 来表明是情况1. 还是情况2. 一般来说, 我们将
这个信号称为 PCsrc. 他表明着 PC 的输入来源. 

\item Register (寄存器)
\label{sec:orgc36f84c}

我们应该有这些功能:
\begin{enumerate}
\item 根据 Register 编号 Rw 将 busW 写入到寄存器之中
\item 根据 Register 编号 Ra Rb 将寄存器的值输出到 busA, busB 上
\end{enumerate}

并且读操作不应收到时钟控制. 其也有控制信号: RegWrite 信号, 表明其是否
要写入. src 有两种可能, 其有可能是来自于内存, 也有可能来自于 ALU. 前者
对应的便是 L 型指令, 后者对应的指令有 I 型指令等. 这种条件的选择也由一
个控制信号来控制, 称为 MemtoReg
\end{enumerate}

\subsubsection{Memory 的简单构成}
\label{sec:org8af6ed2}
\begin{enumerate}
\item 数据存储器
\label{sec:org5c3d388}

应当采用时序逻辑设计. 
其应做到, 将指定的数据 DataIn 写入到 Addr 指定的内存位置里, 并且能够根
据指定的 Addr 将内存中对应的数据写到输出 DataOut 上. 这就是读写操作,
但其中读的操作不应该受到时钟的控制 (至少是可以不受到时钟的控制) 

\item 指令存储器
\label{sec:org37ac6f4}

一个程序运行的之前, 程序装载器将程序装载起来, 在程序运行过程中, 不能对
指令存储器进行写入的操作. 

其应做到
\begin{enumerate}
\item 根据对应的 Addr 给出对应位置存储的指令.
\item 不能在程序运行过程对其进行写操作.
\end{enumerate}
\end{enumerate}


\subsubsection{Control 单元简单介绍}
\label{sec:org4ac250f}
\begin{enumerate}
\item ALU control 单元
\label{sec:org5155788}

说实话我们以及介绍过了. 这里就不介绍了. 

\item Control 单元
\label{sec:orgd7b6fb1}

虽然我们还没怎么说, 但是上文已经提到了非常多的控制信号. 这些控制信号,
比如说 MemtoReg, 比如说 ALUop (ALUop 是作为 ALU control 的控制信号),
这些控制信号, 实际上是直接由指令码的 opcode field 而来, viz., control
单元的输入信号为 instruction[6:0], viz., opcode. 

我们先来数一下有多少个输出信号

\begin{enumerate}
\item Branch 用于分支
\item MemRead 如其名
\item MemtoReg 确定 Reg 的来源
\item ALUop 两位信号, 生成 ALU 的控制信号
\item MemWrite 如其名
\item ALUSrc 确定 ALU 的 source 因为其可以是立即数也可以是寄存器的值.
\item RegRead 如其名
\end{enumerate}

是的, 还真就几把那么多\footnote{PCSrc 是一个衍生信号, 并不是 Control 的直接输出.}. 那么我们可以将 Control 单元看作是一个译
码器:
    I: instruction[6:0]
    O: 上面 8 位数据.
\end{enumerate}

\subsection{在简单指令运行之前}
\label{sec:org0faab94}

在下一个部分开始之前, 我们细说一下各个模块之间的联系. 我们从左到右开始
\subsubsection{PC}
\label{sec:orgd40ef17}

最左边是 PC, PC 有两种情况, PC = PC + 4 以及 PC = PC + offset. 这里的
两种加法不通过 ALU, 而是由两个加法器构成. 其中一个加法器为

PC + 4 

另一个加法器为 

PC + offset

其中 offset 是 imm, 那么就是来自于立即数生成器---imm-Gen 
\subsubsection{IM}
\label{sec:orgbb321d6}

随后是 instruction memory. 输入---PC, 输出---32位的指令---instruction. 

\subsubsection{Reg}
\label{sec:org9aa2fb0}


我们应该有这些功能:
\begin{enumerate}
\item 根据 Register 编号 Rw 将 busW 写入到寄存器之中
\item 根据 Register 编号 Ra Rb 将寄存器的值输出到 busA, busB 上
\end{enumerate}

并且读操作不应收到时钟控制. 其也有控制信号: RegWrite 信号, 表明其是否
要写入. src 有两种可能, 其有可能是来自于内存, 也有可能来自于 ALU. 前者
对应的便是 L 型指令, 后者对应的指令有 I 型指令等. 这种条件的选择也由一
个控制信号来控制, 称为 MemtoReg 
End of quote

能够看出 Ra, Rb, Rw 都是来源于 instruction 的. Register 有可能接
收 1. ALU 的值 2. 内存的值. 

\subsubsection{ALU}
\label{sec:org912a599}

其源可能是 Reg 也可能是 imm. 

PCsrc = Zero and Branch 

对于 bne 指令, 寄存器 A 等于 寄存器 B 的时候 (也就是 A - B = 0) 的时候
进行跳转, 跳转就意味着 PC = PC + offset. 

\subsubsection{Memory}
\label{sec:org9bfd67b}

其输出可能用于 load 指令, load 指令将内存里面的东西放到寄存器里面. 

大概就这些, 一些无伤大雅的复读. 
\subsection{简单指令的运行}
\label{sec:org342a7fe}

以 add rd, rst1, rst2 为例:

\begin{enumerate}
\item PC 取指令, PC + 4
\item ins 的值输入到寄存器组件, rst1 rst2 输入到 ALU 之中
\item ALUop 和 funct 经过 ALU control 中心, 输入给 ALU, 确定 ALU 进行的运
算类型. ALU 得到的值, 输入到寄存器组件之中
\item 寄存器将这个值写到 rd 上.
\end{enumerate}

我们不妨验证一下, Control 的值都是些什么

\begin{center}
\begin{tabular}{rrrrrrr}
ALUSrc & MemtoReg & Regwrite & MemRead & MemWrite & Branch & ALUop\\
0 & 0 & 1 & 0 & 0 & 0 & 10\\
\end{tabular}
\end{center}

ALU 的 sauce 为 rst2; 不设计内存操作; 结果写入 rd; 不是分支判断

\begin{center}
  \begin{tabular}{|l|l|l|l|l|l|} \hline 
funct7 & rst2 & rs1 & funct3 & rd & opcode\\ \hline
\end{tabular}
\end{center}

\noindent\rule{\textwidth}{0.5pt}

以 ld rd offset(rst1) 为例:

\begin{enumerate}
\item PC 取指令, PC + 4
\item 寄存器输出 rst1, imm-Gen 输出 offset, 送入到 ALU 之中
\item ALU 将运算结果送到内存之中, 内存输出对应的值
\item 内存输出的值送到寄存器单元, 写到 rd 上面.
\end{enumerate}

我们进行验证: 
\begin{itemize}
\item ALUsrc   为 1, 因为操作数有 imm
\item MemtoReg 为 1, 因为寄存器将内存的值写到了 rd 上
\item Regwrite 为 1, 因为 rd 被写入了
\item MemRead  为 1, 因为内存要读数据
\item MemWrite 为 0, 因为不用写入内存
\item branch   为 0, 这是肯定的
\end{itemize}

\noindent\rule{\textwidth}{0.5pt}

以 beq rst1, rst2, offset 为例:

\begin{enumerate}
\item PC 取指令
\item 取出 rst1, rst2 的值
\item 取出 offset 的值, 符号扩展, 左移一位, 和 PC 的值进行相加; ALU 进
行 rst1 - rst2 的运算, 输出 Zero
\item 根据 branch 和 zero 的值决定 PC 的值.
\end{enumerate}

可以看见 branch 用上了, 对于控制信号的验证我就不说了. 
\subsection{更多的指令}
\label{sec:orge821a3b}

对于 I 型指令的 jalr 
     J 型指令的 jar 等
     U 型指令的 auipc 等, 
     B 型指令(SB型指令)的 blt, bne 等指令
我们目前没能实现. 比如说 jalr 会将 PC 之中的值存入寄存器中, 我们还没有
     实现这点, auipc 也是同理. blt 则是控制信号还不够, 目前只能有 beq
     的实现. 

实际上这些细节能够自己补全. 1. 接一条线到 ALU 上; 2. PC + offset 的加
法器接出一条线, 接到寄存器上面. 稍微考虑一下, 两种方法都有问题: 1. 接
到 ALU 上, 那么我们在 beq 之中就用不上 ALU 的 Zero 值了, 然后 PC +
offset 的部分都稍微重构一下; 2. 如果说 PC + offset 再接到寄存器单元上,
那么寄存器的输入就有三种情况了, imm, 内存数据, pc 值.  
\end{document}