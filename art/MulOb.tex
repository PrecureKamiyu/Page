\documentclass{ctexart}
\usepackage{amsmath}
\usepackage{newtxtext, newtxmath}

\begin{document}
\section{带权的线性多目标规划法}

\subsection{基本问题}
一句话概括

\begin{equation}
		{\it minimize} \quad \{ \, f _{1} (\mathbf x) , \cdots , f _{k} (\mathbf x) \, \}
\end{equation}
\begin{equation}
		{\it s.t.} \quad
		\begin{cases}
				(\text{约束条件})
		\end{cases}
\end{equation}

\subsection{建模}
一般来说若是出现了尽量, 尽可能等等词语, 则有可能是多目标规划问题. 比如说我建厂, 现在做决策, 我想要
\begin{itemize}
		\item 尽可能使得利润高, 最好超过 10 w.
		\item 尽可能让工人的工作时长接近 12 个小时
\end{itemize}
我们可以从这两句话提取出两个变量, 便是 \(d_{1}^{-}\) 利润和目标利润之间的差值, 以及工作时长和 12 个小时之间的两边的差值 \(d _{2} ^{+}, d_{2} ^{-}\), 满足了 \( d_{2} ^{-} + d_{2} ^{+} + 12 = {\it work \ hour}\). 使得这个问题转换为
\begin{equation*}
		{\it minimize} \quad \{ \, d_{1}^{-}, d _{2} ^{+}, d _{2} ^{-} \, \}
\end{equation*}
当然我还有硬性要求: 工人的工资不得高于 2000 块:
\begin{equation*}
		{\it s.t.} \quad
		\begin{cases}
				1 + d_{1} ^{-} = 1 \\
				d_{2} ^{-} + d_{2} ^{ +} + 12 = {\it work hour} \\
				(\text{总之是和工人工资相关的东西})
		\end{cases}
\end{equation*}

\subsection{线性加权法}

没有权重的线性方法是什么? 即, 令 \(\sum _{i = 1} ^{n} f _{i} (\mathbf x)\) 最小. 若是加权了, 那便是要
\[
1 \big/ \big( \sum _{i = 1} ^{n}  w_{i}\big) \times \big( \sum _{i = 1} ^{n} w _{i} f _{i} (\mathbf x) \big)
\]
最小. And we have
\[
  {\it minimize} \quad 1 \big/ \big( \sum _{i = 1} ^{n}  w_{i}\big) \times \big( w _{1} d _{1} ^- + w_2 d_2 ^{+} + w_3 d_2 ^{-}\big)
\]

\subsection{Solution}
\begin{itemize}
\item Utilize the 序贯函数
\item Use Lingo
\end{itemize}
\section{Ref}
\begin{itemize}
\item https://www.zhihu.com/column/c\_1360363335737843712
\item MultiObject optimization interactive and evolutionary approaches
\end{itemize}

\end{document}